\documentclass[../../rasd.tex]{subfiles}
\begin{document}

\section{Scope\label{sect:1.2}}
	% TODO: rivedere
Crowd-sourced applications have become more popular nowadays thanks to the massive diffusion of smart devices and the consequent interconnection between users so that they started feeling part of a community, where everyone can contribute concretely.\\\

SafeStreets is a crowd-sourced application whose aim is, indeed, to provide a tool to allow citizens to help authorities reporting parking violations that occur in their cities. Examples of parking which represent a violation are the ones on bike lanes, sidewalks, in front of vehicle entrances or on reserved parking lots.\\
Citizens and authorities can also retrieve analytics about data collected by the application, in order to obtain information about violations in certain areas and to identify vehicles with the highest number of violations, respectively.\\\

Moreover, given that a municipality provides access to its data about accidents, the system to be can integrate those information with its own data and finally suggest authorities possible interventions to apply.\\
SafeStreets can also build statistics from data sent by municipalities which concern issued traffic tickets generated by users' reports and make them publicly available.

\subsection{Analysis of Shared Phenomena}
\begin{itemize}
	% TODO: rivedere
\item Users register to SafeStreets; (?)
\item Registered users log into the system; (?)
\item Users take pictures of parking violations;
\item Users fill report fields;
\item Municipality receives data of violation reports;
\item Users access to information provided by SafeStreets;
\item Municipality accesses to information provided by SafeStreets;
\item Municipality sends SafeStreets information about traffic tickets generated from users' reports;
\item Users access to statistics about traffic tickets (built by SafeStreets derived from Municipality feedbacks?);
\item Municipality accesses to statistics about traffic tickets (built by SafeStreets derived from its own feedbacks?).
\end{itemize}

\end{document}