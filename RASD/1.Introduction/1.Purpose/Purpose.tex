\documentclass[../../rasd.tex]{subfiles}
\begin{document}

\section{Purpose\label{sect:1.1}}

This document constitutes the Requirement Analysis and Specification Document (i.e. RASD). Its purpose is to analyze the requirements that will lay the foundations of application services, to specify the application domain, the entities involved and their relationship, to clearly explain the objectives, the constraints and the features that are going to be implemented. \newline
SafeStreets is a crowd-sourced application that intends to provide users with the possibility to notify authorities when traffic violations occur, and in particular parking violations. The application allows users to send pictures of violations, including their date, time, and position, to authorities.
			
			\subsection{Goals\label{sect:1.1.1}}
			\begin{itemize}
				
		\item[G\subs{1}]Allows unregistered user to sign in to access to the application;
		\item[G\subs{2}]Allows registered user to log in and access to the application;
		\item[G\subs{3}]Allows logged in users to take pictures of traffic violations;
		\item[G\subs{4}]Uses an external service to recognize the license plates from pictures obtained by users;
		\item[G\subs{5}]Allows logged in users to select the type of violation;
		\item[G\subs{6}]Collects reports of traffic violations including their pictures, license plate, type of violation, date, time and position;
		\item[G\subs{7}]Sends municipality notifications about traffic violations;
		\item[G\subs{8}]Allows both logged in users and authorities to mine the information that have been received, with different levels of visibility;
		\item[G\subs{9}]Allows logged in users to visualize potentially unsafe areas
		\item[G\subs{10}]Allows logged in users to suggest possible interventions on potentially unsafe areas;
		\item[G\subs{11}]Allows municipality to send feedback about the tickets generated based on users’ reports;
		
			\end{itemize}
\end{document}
