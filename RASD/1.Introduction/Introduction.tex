\providecommand{\rasd}{..}
\documentclass[../RASD.tex]{subfiles}

\begin{document}

\chapter{Introduction}
\thispagestyle{fancy}
		
		\subfile{1.Introduction/1.Purpose/Purpose.tex}
		\subfile{1.Introduction/2.Scope/Scope.tex}
		\subfile{1.Introduction/3.DefinitionsAcronymsAbbreviations/DefinitionsAcronymsAbbreviations.tex}
	
		\section{Revision History}
		\begin{enumerate}
			\item Version 0.1 - 20th October 2019 - Start of the RASD;
			\item Version 1.0 - 10th November 2019 - First release;
			\item Version 1.1 - 15th November 2019
				\paragraph{Major Changes}
				\begin{itemize}
					\item \ic{User reports} State Chart coherent with the specifications 
					\item World and Machine phenomena table added
					\item Use Case Diagrams edited
					\item Alloy comments and chapter description added
				\end{itemize}
				\paragraph{Minor Changes}
				\begin{itemize}
					\item SafeStreets UML Class Diagram
					\item Requirements (+1) and Assumptions (-1)
					\item Use Case Analysis edited
					\item Sequence Diagrams edited
					\item Use Case Scenario edited
				\end{itemize}

		\end{enumerate}
		\section{Reference Documents}
		\begin{itemize}
		\item Rumbaugh, Jacobson, Booch. 1999. \ic{The Unified Modeling Language Reference Manual}. Addison-Wesley.
		\item MIT Software Design Group. \ic{Appendix B: Alloy Language Reference}. \\http://alloytools.org/documentation.html
		\item MIT Software Design Group. \ic{Tutorial Materials, Slides}. \\http://alloytools.org/tutorials/day-course/
		\end{itemize}
		\section{Document Structure}
		This document is structured as follows:
		\paragraph{1. Introduction}
		A general introduction to the goals, the phenomena and the scope of the system-to-be. It aims giving general but exaustive information about what this document is going to explain.
		\paragraph{2. Overall Description}
		A general description of the product to be and its requirements. This section provides several information that are explained in detail in Section 3.
		\paragraph{3. Specific Requirements}
		All software requirements are explained using scenarios, use-case diagram and activity diagram. Non-functional and functional requirements are also cited.
		\paragraph{4. Formal Analysis using Alloy}
		This section includes Alloy code that describes the model and shows its soundness and correctness.
		\paragraph{5. Effort spent}
		Effort spent by all team members shown as the list of all the activities done during the realization of this document.
		\paragraph{6. References}
		References of documents that this project was developed upon.

		
\end{document}