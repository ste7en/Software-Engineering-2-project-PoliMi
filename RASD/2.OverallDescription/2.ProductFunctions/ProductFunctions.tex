\documentclass[../../rasd.tex]{subfiles}
\begin{document}
	
	\section{Product Functions}
	
	\subsection{User registration}
	SafeStreets will allow individuals to register. These will register by entering all the required information (see \hyperref[sect:3.2.5]{R\subs{1}- R\subs{2}}). When registering to SafeStreets, an individual will first declare to have read the \ic{Privacy statement} and secondly, they will have to accept the \ic{Terms and conditions}, which specifically include their consent to the acquisition and processing of their data. \\
	The \ic{User} registration process will be carried out on the SafeStreets application (see section “bho quella che sarà dei mockup”). 
	
	\subsection{User report acquisition}
	SafeStreets will collect \ic{User report}. It will allow \ic{User} to take a picture of the \ic{Traffic violation} and to select the \ic{Type of violation}. Once \ic{License plate recognition} has recognized the license plate, SafeStreets will acquire the \ic{User report} and send it to the \ic{Municipality} after \ic{User} confirmation. 
	
	\subsection{User contribution visualization}
	SafeStreets will show to the \ic{User} his/her contribution in dedicated section where visualize all the previous \ic{User report}. \ic{User} will check also the \ic{Ticket feedback} of each of these, received by the \ic{Municipality}. 
	
	\subsection{Public statistics visualization}
	SafeStreets will show \ic{Public statistics} required by the \ic{User}. \ic{User} will be able to set the parameters selecting a desired “period of time”, a “\ic{Type of violation}” and a ‘’specific zone’’. 
	
	\subsection{Municipality access}
	SafeStreets will allow \ic{Municipality} to connect to its dedicated website.  
	
	\subsection{User report evaluation}
	\ic{Municipality} will visualize the list of \ic{User report} and decide to generate or not a \ic{Traffic ticket}. If \ic{Municipality} discard a \ic{User report}, it will select a reason among those provided.
	
	\subsection{Detailed statistics visualization}
	SafeStreets will show \ic{Detailed Statistics} required by the \ic{Municipality}. \ic{Municipality} will be able to set the parameters selecting the “period of time”, a “\ic{Type of violation}”, a ‘’specific zone’’ and also inserting a ‘’specific license plate’’. 
	
	\subsection{Possible interventions visualization}
	SafeStreets will provide to cross \ic{Municipality} \ic{Accidents} data with its own Database data. Based on the results, it will suggest \ic{Possible interventions} to \ic{Municipality}. 
	
	\subsection{License plate recognition data acquisition}
	The service will be able to receive all the picture taken by \ic{Users} and sent through SafeStreets. Once recognized the license plate, they will return the text that represents it. 
	
	\subsection{Request Accidents data}
	SafeStreets, requesting access to \ic{Municipality} \ic{Accidents} Database, will able to cross \ic{Municipality} data with its own data. 
	
	\subsection{Request User report data}
	\ic{User}, requesting access to SafeStreets database for his/her own \ic{User report}, will be able to visualize all details about his/her previous \ic{User report}. 
	
\end{document}