\documentclass[../../../rasd.tex]{subfiles}
\begin{document}

\subsection{Scenarios\label{sect:3.2.1}}

\begin{itemize}
	\item[S\subs{1}]Michael is an individual who would like to help the community and the authorities reporting parking violations occurring in his city. He downloads the SafeStreets application on his smartphone and, once opened it, he is asked to insert an email (which will correspond to his \ic{User identifier}) and a password. After inserting them, he accepts the \ic{Terms and conditions} and taps on the arrow below. Later, he tries to login using his email and password; the system accepts his credentials and Michael is now logged in;
	
	\item[S\subs{2}]A \ic{Municipality} officier opens SafeStreets web app. He is asked to insert \ic{Reference code} and password in order to access the system, so he inserts his \ic{Municipality} institutional credentials and he is now logged in and can use SafeStreets services;
	
	\item[S\subs{3}]Michael, a SafeStreets \ic{User}, is walking down a street and, when he’s about to cross the road, he realizes he can’t access the pedestrian crossing from the walk side, because of a car parked there. So he opens the app and presses the “Camera” button in the tab bar of the main view, frames the car, making sure that the violation and the license plate are clearly visible, and takes the picture. After pressing the “Confirm” button he is asked to select the “\ic{Type of violation}”, choosing from a drop-down list of possible violations, and presses again on the “Send” button, in order to send his \ic{User report};
	
	\item[S\subs{4}]Michael, from scenario S\subs{3}, opens the app and taps on “MyReports” in the tab bar of the main view, to see if the last \ic{User report} he has sent has already been evaluated, eventually becoming a \ic{Traffic ticket}, but he sees that the \ic{Ticket feedback} color is red, which means that a \ic{Traffic ticket} has not been generated from his \ic{User report}. To know more, he presses on the “More info” button and understands that the reason why it was not generated is that it was not an actual violation;
	
	\item[S\subs{5}]Scarlett, a SafeStreets \ic{User}, wants to visualize \ic{Public statistics} of her neighborhood of the last month, in order to figure out whether her neighbours have been active in reporting \ic{Traffic violation} through SafeStreets or not, and eventually pointing it out during the monthly neighborhood meeting. For that reason, she presses the “\ic{Public statistics}” button and selects an area, a period and a \ic{Type of violation} from the corresponding drop-down lists and consults the statistics she is interested in;
	
	\item[S\subs{6}]\ic{Municipality} receives the complaint for a stolen car, so, in order to increase the possibilities to find some useful clues, looks for \ic{User reports} received in the last 24 hours and involving the stolen car license plate in the \ic{Detailed statistics};
	
	\item[S\subs{7}]SafeStreets retrieves previous \ic{Accidents} data from \ic{Municipality} Database and, after crossing them with \ic{User report} data in its own Database, identifies the area where the sum of the number of \ic{Accidents} and the number of \ic{User report}, for a particular \ic{Type of violation}, is the highest, in order to identify and suggest a \ic{Possible intervention} to the \ic{Municipality};
	
	\item[S\subs{8}]\ic{Municipality} wants to visualize \ic{Ticket feedback} statistics in the \ic{Detailed statistics} to point out the areas where the service has had the highest incidence (i.e. the areas where the highest number of \ic{User reports} have actually become \ic{Traffic tickets}), in order to intensify the presence of Local Police agents in those areas;
	
	\item[S\subs{9}]\ic{Municipality} wants to visualize who have been the most egregious offenders in a certain area of the city, so looks for them in the \ic{Detailed statistics} focusing in particular on the \ic{Ticket feedback} statistics.
	
\end{itemize}

\end{document}