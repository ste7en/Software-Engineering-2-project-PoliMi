\documentclass[../DD.tex]{subfiles}

\begin{document}

\chapter{Introduction}
\thispagestyle{fancy}
		\subfile{1.Introduction/1.Purpose/Purpose.tex}
		\subfile{1.Introduction/2.Scope/Scope.tex}
		\subfile{1.Introduction/3.DefinitionsAcronymsAbbreviations/DefinitionsAcronymsAbbreviations.tex}
		
		\section{Revision History}
		\begin{enumerate}
			\item Version 0.1 - 11th November 2019 - Start of the DD;
			
			%\item Version 1.0 - Date - First Release
			%	\paragraph{Major Changes}
			%	\begin{itemize}
			%		\item Description + sections.
			%	\end{itemize}
			
		\end{enumerate}
		\section{Reference Documents}
		\begin{itemize}
			\item Rumbaugh, Jacobson, Booch. 1999. \ic{The Unified Modeling Language Reference Manual}. Addison-Wesley.
		\end{itemize}
		\section{Document Structure}
		This document is structured as follows:
		\paragraph{1. Introduction}
		A general introduction to the system-to-be. It aims giving general but exhaustive information about what this document is going to explain.
		\paragraph{2. Architectural Design}
		An overview about the high-level components	and	their interaction, with focus on both static and dynamic view, helped by diagrams.
		\paragraph{3. User Interface Design}
		A representation of how the user interfaces will look like.
		\paragraph{4. Requirements Traceability}
		An explaination about how the requirements defined in the RASD map to the design elements defined in this document.
		\paragraph{5. Implementation, Integration and Test Plan}
		Identification of the order in which the subcomponents of the system should be implemented and of the order in which they should be integrated and tested.
		\paragraph{6. Effort spent}
		Effort spent by all team members shown as the list of all the activities done during the realization of this document.
		\paragraph{7. References}
		References of documents opun which this project was developed.
		
		
		
\end{document}