\documentclass[../../DD.tex]{subfiles}
\begin{document}

\section{Scope\label{sect:1.2}}
	
Crowd-sourced applications have become more popular nowadays thanks to the massive diffusion of smart devices and the consequent interconnection between people so that they started feeling part of a community, where everyone can contribute concretely.\\

SafeStreets is a crowd-sourced application whose aim is, indeed, to provide a tool to allow registered citizens to help \ic{Municipality}, reporting \ic{Traffic violations} that occur in their cities. In particular, they can take pictures of parking violations, specifying their type. A \ic{License plate recognition service} recognizes the license plate from the \ic{User picture}. SafeStreets stores the generated \ic{User reports} and send a \ic{User report notification} to \ic{Municipality}. Examples of parking which represent a violation are the ones on bike lanes, sidewalks, in front of vehicle entrances or on reserved parking lots.\\
\ic{Users} and \ic{Municipalities} can also retrieve analytics about data collected by the application in order to obtain information, for example, about violations in certain areas or to identify vehicles with the highest number of violations, respectively.\\

Moreover, given that \ic{Municipality} provides access to its data about \ic{Accidents}, the system-to-be can integrate those information with its own data and finally suggest \ic{Municipality} \ic{Possible interventions} to apply.\\
SafeStreets can also build statistics from data sent by \ic{Municipality} which concern issued \ic{Traffic tickets} generated by \ic{User reports}, for example about the most egregious offenders, or to show the effectivness of the SafeStreets initiative.

\end{document}