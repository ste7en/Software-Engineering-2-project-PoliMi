\documentclass[../../DD.tex]{subfiles}
\begin{document}

\section{Other design decisions\label{sect:2.7}}

\subsection{User Authentication and Password Storing\label{2.7.1}}
In order to guarantee the confidentiality of \ic{Users'} and \ic{Municipalities'} data, the access to the system is guaranteed by entering a personal identifier and a password. For \ic{Users}, the personal identifier coincides with the email address inserted during the registration phase. \\

For each registered entity, their access passwords are hashed using a SHA512 algorithm and then saved in the relative Database.

\subsection{Chain of Custody\label{2.7.2}}
\ic{User reports} constitutes sensitive data because they contain license plates' numbers, GPS data and, in general, information about \ic{Traffic violations} that will be shown to a \ic{Municipality}. To ensure that these information won't be altered, a digital signature mechanism has been implemented to realize the chain of custody. \\

In particular, when a \ic{User} sends his/her \ic{User report} to the Application Server, a hash function on its data and the \ic{User} entity is computed, returning the digital signature corresponding to the specific report. This hashcode is, then, stored in the Authentication Database and accessed at retrieval time. 
When a \ic{Municipality} client sends a request to access those \ic{User reports}, the same hash function is computed and the result is checked with the stored one, in order to ensure that no alterations have occurred. \\



\subsection{Relational Database\label{2.7.3}}
The data management of the system will be handled by a Relational DBMS. This choice has, from the organization point of view, both internal and external motivations: the first thing to consider is the effort and time spent by the development team, which is familiar with relational databases, while the usage of a NoSQL DB would require time for the study and design process of it.
Moreover, a relational database is transactional, each access operation represents a real-world event of every enterprise and the so-called ACID properties are guaranteed:

	\begin{description}
	\item[Atomicity]: no partial execution is allowed;
	\item[Consistency]: every state is consistent after a transaction execution;
	\item[Isolation]: transactions should be executed in isolation from other transactions;
	\item[Durability]: even in case of failures, changes in the database should persist.
	\end{description}

\end{document}